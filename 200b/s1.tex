\documentclass[12pt]{article}
\usepackage{amsmath}
\usepackage{amssymb}

\author{Alex Zahn}
\title{Phys 200b Problem Set I}
\date{}

\begin{document}

\maketitle

\newcommand{\tb}{\overline{\theta}}
\newcommand{\tti}{\tilde{\theta}}
\newcommand{\dtb}{\dot{\overline{\theta}}}
\newcommand{\dtti}{\dot{\tilde{\theta}}}
\newcommand{\ddtb}{\ddot{\overline{\theta}}}
\newcommand{\ddtti}{\ddot{\tilde{\theta}}}

\newcommand{\pb}{\overline{\varphi}}
\newcommand{\pti}{\tilde{\varphi}}
\newcommand{\dpb}{\dot{\overline{\varphi}}}
\newcommand{\dpti}{\dot{\tilde{\varphi}}}
\newcommand{\ddpb}{\ddot{\overline{\varphi}}}
\newcommand{\ddpti}{\ddot{\tilde{\varphi}}}

\newcommand{\coswt}{\cos\omega t}
\newcommand{\sinwt}{\sin\omega t}

\newcommand{\wn}{\omega_0}

\newcommand{\average}[2]{ \left\langle\ #1 \right\rangle_#2  }

\newcommand{\ueff}{u_{\textit{eff}}}

\subsection*{1a.}

Knowing \(x=l\cos\theta + x_{0}\cos\omega t\) and \(y = l(1-\cos\theta)\), we can compute a Lagrangian that goes over straightforwardly into the equation of motion,

\[\ddot{\theta} = -\frac{g}{l}\sin\theta + \frac{\omega^2x_0}{l}\cos\theta\cos\omega t
\]

Assume \(\theta\approx\tb+\tti\), where \(\tti\) varies on time scales of \(2\pi/\omega\) and is higher order than \(\tb\), which should vary on timescales closer to the natural period of the undriven system \( 2\pi/\wn\), \( \wn^2=g/l  \). \(\tb\) then carries the average motion of the system, with \(\tti\) being a fast perturbation. We further assume both are periodic.

Note that this is completely different than the more familiar perturbation expansion approach, where we would attempt \(\theta \approx \theta_0 + \theta_1\), where \(\theta_0\) is the solution to the undriven problem and \( \theta _1\) is a first order correction term. \(\tb\) shares nothing in common with \(\theta_0\), and will be seen to follow a very different equation of motion that depends on our results for \(\tti\). The only similarity here is that both \(\tti\) and \(\theta_1\) are lower order than \(\tb\) and \(\theta_0\), respectively.

So substituting \(\tb+\tti\) for \(\theta\) in the equation of motion and expanding to first order in \(\tti\),

\begin{equation} \label{eq:1}
\ddtb + \ddtti = -\frac{g}{l}\sin\theta  - \frac{g}{l}\tti\cos\tb + \frac{x_0\omega^2}{l}\coswt\cos\tb - \frac{x_0\omega^2}{l}\tti\coswt\sin\tb
\end{equation}

We can average both sides over a period \(T\) of \( \tti \). Apparently it's invalid to attempt the same average over a period \(T_0\) of \(\tb\), and if anyone reading this has any ideas about why, I would be much obliged to hear them.

\begin{equation}\label{eqn:2}
\average{\ddtb + \ddtti}{T} = \average{-\frac{g}{l}\sin\theta  - \frac{g}{l}\tti\cos\tb + \frac{x_0\omega^2}{l}\coswt\cos\tb -  \frac{x_0\omega^2}{l}\tti\coswt\sin\tb}{T}
\end{equation}

All of the fast varying terms that have periods near \(T\) become negligible, or drop out exactly (such as \(\average{\ddtti}{T}\), which has period \(T\) by construction), and the above becomes 

\begin{equation}\label{eqn:3}\average{\ddtb}{T} = \average{-\frac{g}{l}\sin\theta}{T} - \average{\frac{x_0\omega^2}{l}\tti\coswt\sin\tb}{T}
\end{equation}

Over timescales of \(T\), \(\ddtb\) is effectively constant and we have 

\begin{equation}\label{eqn:4} \ddtb = -\frac{g}{l}\sin\theta - \frac{x_0\omega^2}{l}\sin\tb\average{\tti\coswt}{T}
\end{equation}

Apparently, we can also construct an equation from equality of the fast terms in \ref{eqn:2}:

\begin{equation}\label{eqn:5}
\ddtti = -\frac{g}{l}\cos\tb + \frac{\omega^2x_0}{l}\cos\tb\coswt
\end{equation}

I don't entirely understand this step. First, I would have thought that there is a fast component to the \(\tti\coswt\) term from \ref{eqn:4}, which should have a frequency near \(2\omega\), and that it should contribute to \ref{eqn:5}. Second, I'm not sure exactly why it's valid to equate all the vanishing or nearly vanishing terms like this (maybe there is some sort of linear independence argument?).

Moving on, we assume the first term is negligible since \(g/l \ll \omega^2x_0/l\), and hypothesize a solution for \(\tti\) of the form \(a(\tb)\coswt + b(\tb)\sinwt\). Discarding terms of order \(\dtb\), this yields

\[\tti = -\frac{x_0}{l}\cos\tb \coswt
\]

Substituting into \ref{eqn:5},

\begin{align*}
\ddtb = -\frac{g}{l}\sin\theta - \frac{x_0^2\omega^2}{2l^2}\cos\tb\sin\tb
\end{align*}

Then the effective potential defined by \(\ddtb \equiv -\frac{d}{d\theta}\ueff \) is

\[\ueff = -\frac{g}{l}\cos\tb + \frac{x_0^2 \omega^2}{4l^2}\sin^2 \tb
\]

It follows that \(\tb = 0\) is a stable equilibrium if \(g/l > \frac{x_0^2\omega^2}{2l}\) and \(\tb=\pi\) is stable if \(\tb=0\) is unstable.
\\

Unrelated, I don't know how to get \(\average{\ddtti}{T}\) to render correctly.

\subsection*{1b.}

This time \(x = l\cos\theta + r_0\coswt\), \(y = l(1-\cos\theta) + r_0\sinwt \) and we find the Euler-Lagrange equation with some computer assisted algebra:

\begin{align*}\ddot{\theta} &= \frac{r_0\omega^2}{l}\left(  \cos\theta \coswt - \sin\theta \sinwt  \right) - \frac{g}{l}\sin\theta \\
\end{align*}

Proceeding as above, 

\[\ddtb + \ddtti = \frac{r_0\omega^2}{l}(\sin\tb\sinwt + \tti\cos\tb\sinwt + \cos\tb\coswt ) - \frac{g}{l}(\sin\\tb - \tti\cos\tb)
\]

Time averaging both sides and taking anything with frequency \(\omega_0\) to be constant,

\[\ddtb = \frac{r_0\omega^2}{l}\left(\cos\tb\average{\tti\sinwt}{T}-\sin\tb\average{\tti\coswt}{T}\right)
\]

Once again equating the fast terms, we also have

\[\ddtti=\frac{r_0\omega^2}{l}(\sin\tb\sinwt + \cos\tb\coswt) - \frac{g}{l}\tti\cos\tb
\]

Attempting a solution of the form \( a(\tb)\sinwt + b(\tb)\coswt\) and dropping the \(g/l\) term as before,

\[\tti =  \frac{-r_0}{l}(\sin\tb\sinwt + \cos\tb\coswt)
\]

Substituting into the \(\ddtb\) equation and dropping averaged terms with frequency \(\omega\) and \(2\omega\),

\[\ddtb = -\frac{g}{l}\sin\tb
\]

So we recover the simple harmonic oscillator, and we just have the usual stable equilibrium at \(\tb = 0\).

\pagebreak

\subsection*{2.}

We have the equation of motion

\[\ddot{\theta} = \frac{g}{l}\sin\theta + \frac{y_0\omega^2}{l}\sin\theta\coswt
\]

The idea is try to transform this into the Mathieu  Equation, and then apply the textbook recipe to it for analyzing its parametric instability. The Mathieu Equation is only first order in \(\theta\), so we try expanding the equation of motion to first order in \(\theta\):

\[\ddot{\theta} = \frac{g}{l}\theta + \frac{y_0\omega^2}{l}\theta\coswt
\]

Then with \(h\equiv \frac{4y_0}{l}\) and \(\omega^2 = (2\omega_0+\epsilon)^2 \approx 4\omega_0^2\),

\[\ddot{\theta} + \omega_0^2(1+h\coswt)\theta = 0
\]

\newcommand{\coswne}{\cos(\omega_0 + \epsilon/2)}
\newcommand{\sinwne}{\sin(\omega_0 + \epsilon/2)}

which is the form we need. The usual recipe for dealing with this thing is to attempt a solution of the form \(\theta = a(t)\coswne + b(t)\sinwne \) where \(\ddot{a}\approx\ddot{b}\approx0\). This  yields

\[ \sinwne(-2\omega_0\dot{a} - b\omega_0\epsilon - \omega_0^2hb/2) + \coswne(2\omega_0\dot{b}-a\epsilon\omega_0+\omega_0^2ha/2) = 0
\]

The second step in the recipe is to propose \(a = a_0e^{st}\), \(b = b_0e^{st}\). Using linear independence of \(\sinwne\) and \(\coswne\),

\begin{align*}
-2\omega_0\dot{a} - b\omega_0\epsilon - \omega_0^2hb/2 &= 0 \\
2\omega_0\dot{b}-a\epsilon\omega_0+\omega_0^2ha/2 &= 0 \\
\end{align*}

Substituting  \(a = a_0e^{st}\), \(b = b_0e^{st}\),

\begin{align*}
sb_0 &= \left( \frac{\epsilon}{2}-\frac{\wn h}{4} \right)a_0\\
sa_0 &= -\left( \frac{\epsilon}{2}+\frac{\wn h}{4} \right)b_0
\end{align*}

Multiplying both equations together, 

\[s = \sqrt{\frac{\wn^2 h^2}{16}-\frac{\epsilon^2}{4}}
\]

If this is a positive real, we have an instability with growth rate \(s\).

\subsection*{3.}

This time the equation of motion is

\[ \ddot{\phi} + 2\gamma\dot{\phi} + \wn^2\phi(1+h\coswt)=0
\]

with \(h = 4y_0/l\) as before. Again, we attempt a solution of the form \(a\coswne + b\sinwne\), but this time, since we're only interested in the threshold for instability, we only want the point at which the growth rate is zero and \(a\) and \(b\) are constants. The equation of motion reduces to

\newcommand{\wne}{(\wn+\epsilon /2)}

\begin{align*}
-a\wne ^2\coswne &-b\wne^2\sinwne -2\gamma a\wne\sinwne \\
  &+2\gamma b\wne\coswne+a\wn^2\coswne \\
  &+at\wn^2 h\cos (2\wn + \epsilon)\coswne \\
  &+ bt\wn^2 h\cos (2\wn + \epsilon)\sinwne + b\wn^2 \sinwne \\
  &=0
\end{align*}

Dropping terms of order \(\epsilon^2\) we have after a few cancellations,

\[(-a\wn\epsilon+2\gamma b\wn + a\wn^2h/2)\coswne + (-b\wn\epsilon - 2\gamma a\wn + b\wn^2h/2)\sinwne=0
\]

Again noting the linear independence of \(\coswne\) and \(\sinwne\),

\begin{align*}
-a\wn\epsilon+2\gamma b\wn + a\wn^2h/2 &= 0 \\
-b\wn\epsilon - 2\gamma a\wn + b\wn^2h/2 &= 0 
\end{align*}

The above holds if \(\epsilon = \sqrt{\frac{\wn^2h^2}{4}-4h^2}\), so we infer that if the mismatch parameter is greater than this value, we have an instability. It's also worth noting that if \(y_0 < \gamma l/\wn\), there cannot exist an instability for any value of \(\epsilon\).

\newpage

\subsection*{4a.}

We are considering the Hamiltonian 
\[H = \frac{p^2}{2m} + V_0(q) + V(q)\ddot{A}
\]

where \(p_0\), \(q_0\) have period \(T_0\) and result from \(H_0\), and \(A\) is periodic with period \(T\ll T_0\). Hamilton's equations are then

\begin{align*}
\dot{p} = -&\frac{\partial H}{\partial q} = -\frac{\partial V_0}{\partial q} -\ddot{A}\frac{\partial V}{\partial q} \\
\dot{q} = &\frac{\partial H}{\partial p} = \frac{p}{m}
\end{align*}

So \(\ddot{q} = \dot{p}/m\), which expands to

\[ \ddot{q} = \frac{1}{m}\left( -\frac{\partial V_0}{\partial q} -\ddot{A}\frac{\partial V}{\partial q} \right)
\]

\newcommand{\ddb}[1]{\ddot{\bar{#1}}}
\newcommand{\db}[1]{\dot{\bar{#1}}}
\newcommand{\ddti}[1]{\ddot{\tilde{#1}}}
\newcommand{\dti}[1]{\dot{\tilde{#1}}}

Dividing \(q\) into a slow component \(\bar{q}\) that varies on timescales \(T_0\) and a fast component \(\tilde{q}\) that varies on timescales \(T\), we substitute \(q = \bar{q}+\tilde{q}\) into the above and expand to first order in \(\tilde{q}\):

\[ m(\ddb{q} + \ddti{q}) =  -\left(\frac{\partial V_0(\bar{q})}{\partial q} + \tilde{q}\frac{\partial^2 V_0(\bar{q})}{\partial q^2}\right) - \ddot{A}\left(\frac{\partial V(\bar{q})}{\partial q} + \tilde{q}\frac{\partial^2 V(\bar{q})}{\partial q^2}\right)
\]

In these problems, terms are classified as quickly or slowly varying as if this were similar to a parity relation:

\begin{align*}
fast \times slow &= fast \\
fast \times fast &= slow \\
slow \times slow &= slow
\end{align*}

As mentioned in the first problem, I have issues with this because of the identity

\[ \cos(\omega_1 t)\cos(\omega_2 t) = \frac{1}{2}(\cos(\omega_1 + \omega_2) + \cos(\omega_1 - \omega_2))
\]

which would seem to contradict the above. Digression aside, we can average over a period \(T\), which should eliminate or render negligible the fast terms that vary on timescales of order \(T\). Noticing that the \(A\) is fast and that \(V(\bar{q})\) and \(V_0(\bar{q})\) are slow,

\[ \average{m\ddb{q}}{T} = \average{-\frac{\partial V_0(\bar{q})}{\partial q} - \ddot{A}\tilde{q}\frac{\partial^2 V(\bar{q})}{\partial q^2} }{T}
\]

Treating terms that vary over \(T_0\) as constant,

\[m\ddb{q} = -\frac{\partial V_0(\bar{q})}{\partial q} - \frac{\partial^2 V(\bar{q})}{\partial q^2}\average{\ddot{A}\tilde{q}}{T}
\]

This would be the mean field equation we're looking for if we knew \(\tilde{q}\). Taking for granted equality of the fast terms,

\[m\ddot{\tilde{q}} = - \tilde{q}\frac{\partial^2 V_0(\bar{q})}{\partial q ^2} - \ddot{A}\frac{\partial V(\bar{q})}{\partial q}
\]

We solve this by arguing that the \( - \tilde{q}\frac{\partial^2 V_0(\bar{q})}{\partial q ^2} \) term is negligible: if \(\wn\) is the angular frequency of \(\bar{q}\), then \(\frac{\partial^2 V_0(\bar{q})}{\partial q ^2} \) is of order \(\wn^2\) while \( \ddti{q} \) is of order \(\omega^2\). Then

\[ m\ddot{\tilde{q}} = - \ddot{A}\frac{\partial V(\bar{q})}{\partial q}
\]

\(V\) is independent of time by construction, so

\[ \tilde{q} = -\frac{A}{m}\frac{\partial V(\bar{q})}{\partial q}
\] 

where we have dropped the integration constant, though I don't know why it's valid to do so. Returning to the mean field equation, all that's left to do is compute

\begin{align*}
\average{\ddot{A}\tilde{q}}{T} &= \frac{-1}{m}\frac{\partial V(\bar{q})}{\partial q}\average{A\ddot{A}}{T} \\
&= \frac{-1}{m}\frac{\partial V(\bar{q})}{\partial q}\average{ \frac{d}{dt}(A \dot{A}) - \dot{A}^2  }{T}
\end{align*} 

Recalling that \(H\) is only dependent on \( \ddot{A} \) and not on \(A\) itself, we are free to let \(A\) be zero-mean without loss of generality. Then since \(A\) is \(T\)-periodic we have \( \average{A\dot{A}}{T} = 0\) and the mean field equation finally becomes

\[ m\ddb{q} = -\frac{\partial V_0(\bar{q})}{\partial q} - \frac{1}{m}\frac{\partial V(\bar{q})}{\partial q}\frac{\partial^2 V(\bar{q})}{\partial q^2}\average{\dot{A}^2}{T}
\]

\subsection*{4b.}

The effective Hamiltonian 

\[K = \frac{p^2}{2m} + V_0 + \frac{1}{4m}\average{\dot{A}^2}{T}\left( \frac{\partial V}{\partial q}\right)^2
\]

yields

\begin{align*}
\dot{p} &= -\frac{1}{2m}\average{\dot{A}^2}{T}\frac{\partial V}{\partial q}\frac{\partial^2 V}{\partial q^2}\\
\dot{q} &= \frac{p}{m}
\end{align*}

so exactly as before, \(\ddot{q} = \dot{p}/m\) and we recover the mean field equation from above right away.

\subsection*{5a.}

We start with the general statement in an intertial frame for the net external torque on rigid body

\[N^{\textit{ext}} = \left(\frac{dL}{dt}\right)_\textit{intertial}
\]

Transforming in the usual way to body fixed coordinates,

\[ N^{\textit{ext}} = \left( \frac{dL}{dt}\right)_\textit{body} + \omega \times L
\]

Since \(I\omega = L\) this is just


\[N^{\textit{ext}} = I\dot{\omega} + \omega \times I\omega
\]

The moment of intertia tensor should be diagonal in this frame so that expands to the Euler equations:

\begin{align*}
I_1\dot{\omega_1} &= (I_2-I_3)\omega_2 \omega_3 + N^{\textit{ext}}_1 \\
I_2\dot{\omega_2} &= (I_3-I_1)\omega_3 \omega_1 + N^{\textit{ext}}_2 \\
I_3\dot{\omega_3} &= (I_1-I_2)\omega_1 \omega_2 + N^{\textit{ext}}_3
\end{align*}

\subsection*{5b.}

First consider the case where \(\omega_2 = \omega_0 + \epsilon_2\), \(\omega_1 = \epsilon_1\), and \(\omega_3 = \epsilon_3\). There are no external torques, so the Euler equations become

\begin{align*}
I_1\dot{\epsilon_1} &= (I_2-I_3)(\omega_0+\epsilon_2) \epsilon_3 \\
I_2\dot{\epsilon_2} &= (I_3-I_1)\epsilon_3 \epsilon_1 \\
I_3\dot{\epsilon_3} &= (I_1-I_2)\epsilon_1 (\omega_0 + \epsilon_2) 
\end{align*}

Going to first order in \(\epsilon_i\),

\begin{align*}
I_1\dot{\epsilon_1} &= (I_2-I_3)\omega_0 \epsilon_3 \\
I_2\dot{\epsilon_2} &= 0 \\
I_3\dot{\epsilon_3} &= (I_1-I_2)\omega_0 \epsilon_1
\end{align*}

Differentiating the last line,

\[ I_3 \ddot{\epsilon_3} = (I_1-I_2)\omega_0 \dot{\epsilon_1}
\]

We already have an expression for \(\epsilon_1\) with the first Euler equation, so substituting yields

\[ \ddot{\epsilon_3} = \frac{1}{I_1I_3}(I_1-I_2)(I_2-I_3)\wn^2 \epsilon_3
\]

The solution is \( \epsilon_3 = e^{st} \) where

\[ s = \wn \sqrt{\frac{(I_1-I_2)(I_2-I_3)}{I_1I_3}} \equiv \wn\sqrt{\alpha}
\]

Instability results if \(\alpha\) is positive, which we can determine from knowing that \(I_1<I_2<I_3\). Here \(\alpha\) is clearly positive, so we've shown that we have an instability.

The other two cases follow from cyclic permutation of the above. We have

\begin{align*}
\alpha_{\epsilon_1} &= \frac{(I_2-I_3)(I_3-I_1)}{I_2I_1} < 0\\
\alpha_{\epsilon_2} &= \frac{(I_3-I_1)(I_1-I_2)}{I_3I_2} < 0
\end{align*}

So those cases are stable.

\subsection*{5c.}

The immediately obvious conserved quantities are the magnitude of the angular momentum and the energy.





\end{document}