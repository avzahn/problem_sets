\documentclass[12pt]{article}
\usepackage{amsmath}

\begin{document}

\subsection*{1.}

This appears to be a Compton scattering problem. Conserving energy before and after the collision, we have

\[E_{in} + Mc^2 = E_{out} + E_e
\]

\noindent where \(E_{e}\) is the momentum of the charged particle after the collision. 3-momentum is also conserved:

\[E_{in}/c = E_{out}/c + p_{e}
\]

\noindent The scattered photon is deflected by \(\theta\), so the law of cosines gives

\[p_{e}^2 = E_{out}^2/c^2 + E_{in}^2/c^2 - 2E_{in}E_{out}\cos{\theta}/c^2
\]

\noindent Additionally, we have the usual energy-momentum relation for the charged particle:

\[p_{e}^2c^2 = E_{e}^2 - M^2c^4\]

\noindent Substituting for \(E_e\) from the energy conservation equation,

\begin{align*}
p_e^2c^2 &= (E_{in} - E_{out} + Mc^2)^2 - M^2c^4 \\
&= (E_{in} - E_{out})^2 + 2(E_{in} - E_{out})Mc^2
\end{align*}

\noindent Substituting for \(p_e^2\):

\begin{align*}
E_{out}^2 + E_{in}^2 - 2E_{in}E_{out}cos\theta = (E_{in} - E_{out})^2 + 2(E_{in} - E_{out})Mc^2
\end{align*}

\noindent This simplifies to

\[Mc^2(E_{in}-E_{out}) = E_{in}E_{out}(1-cos\theta)
\]
\pagebreak

\subsection*{2.}

\noindent The forces on the oscillator are 

\begin{align*}
F_B &= e\dot{r}\times B\hat{z}/c \\
F_\omega  &= -\omega_0^2r
\end{align*}

\noindent This yields the coupled system

\begin{align*}
m\ddot{x} + \omega_0^2x - eB\dot{y}/c &= 0 \\
m\ddot{y} + \omega_0^2y - eB\dot{x}/c &= 0 \\
m\ddot{z} + \omega_0^2z &= 0
\end{align*}

\noindent Defining \(\xi = x + iy\) and adding \(i\) times the second equation to the first as the solution does,

\[\ddot{\xi}+\omega_0^2\xi = -i\frac{eB}{mc}\dot{\xi}
\] 
\noindent Attempting a solution of the form \( Ae^{i\omega t}\),

\[-\omega^2 + \omega_0^2 = \frac{eB}{mc}\omega
\]

\noindent whence we recover Landau and Lifshitzs' formula for the oscillation frequency in the \(xy\) plane (the oscillation frequency along the \(z\)-axis is obviously still \(\omega_0\)):

\[\omega = \sqrt{\omega_0^2 +\frac{1}{4}\left(\frac{cB}{mc}\right)^2} \pm \frac{eB}{2mc}
\]

\subsection*{3}

\noindent We have

\begin{align*}
\left(\begin{matrix}
E_\gamma/c \\
E_\gamma/c \\
0 \\
0
\end{matrix}\right) +
\left(\begin{matrix}
Mc \\
0 \\
0 \\
0
\end{matrix}\right) =
\left(\begin{matrix}
E_\gamma' \\
0 \\
0 \\
0
\end{matrix}\right)
\end{align*}

\end{document}