\documentclass[12pt]{article}
\usepackage{amsmath}

\author{Alex Zahn}
\title{Phys 203a Problem Set I}
\date{}


\begin{document}

\maketitle

\subsection*{1.}

This appears to be a Compton scattering problem. Conserving energy before and after the collision, we have

\[E_{in} + Mc^2 = E_{out} + E_e
\]

\noindent where \(E_{e}\) is the momentum of the charged particle after the collision. 3-momentum is also conserved:

\[E_{in}/c = E_{out}/c + p_{e}
\]

\noindent The scattered photon is deflected by \(\theta\), so the law of cosines gives

\[p_{e}^2 = E_{out}^2/c^2 + E_{in}^2/c^2 - 2E_{in}E_{out}\cos{\theta}/c^2
\]

\noindent Additionally, we have the usual energy-momentum relation for the charged particle:

\[p_{e}^2c^2 = E_{e}^2 - M^2c^4\]

\noindent Substituting for \(E_e\) from the energy conservation equation,

\begin{align*}
p_e^2c^2 &= (E_{in} - E_{out} + Mc^2)^2 - M^2c^4 \\
&= (E_{in} - E_{out})^2 + 2(E_{in} - E_{out})Mc^2
\end{align*}

\noindent Substituting for \(p_e^2\):

\begin{align*}
E_{out}^2 + E_{in}^2 - 2E_{in}E_{out}cos\theta = (E_{in} - E_{out})^2 + 2(E_{in} - E_{out})Mc^2
\end{align*}

\noindent This simplifies to

\[Mc^2(E_{in}-E_{out}) = E_{in}E_{out}(1-cos\theta)
\]


\subsection*{2.}

\noindent The forces on the oscillator are 

\begin{align*}
F_B &= e\dot{r}\times B\hat{z}/c \\
F_\omega  &= -\omega_0^2r
\end{align*}

\noindent This yields the coupled system

\begin{align*}
m\ddot{x} + \omega_0^2x - eB\dot{y}/c &= 0 \\
m\ddot{y} + \omega_0^2y - eB\dot{x}/c &= 0 \\
m\ddot{z} + \omega_0^2z &= 0
\end{align*}

\noindent Defining \(\xi = x + iy\) and adding \(i\) times the second equation to the first as the solution does,

\[\ddot{\xi}+\omega_0^2\xi = -i\frac{eB}{mc}\dot{\xi}
\] 
\noindent Attempting a solution of the form \( Ae^{i\omega t}\),

\[-\omega^2 + \omega_0^2 = \frac{eB}{mc}\omega
\]

\noindent whence we recover Landau and Lifshitzs' formula for the oscillation frequency in the \(xy\) plane (the oscillation frequency along the \(z\)-axis is obviously still \(\omega_0\)):

\[\omega = \sqrt{\omega_0^2 +\frac{1}{4}\left(\frac{cB}{mc}\right)^2} \pm \frac{eB}{2mc}
\]

\pagebreak


\subsection*{3}

\noindent The idea is to transform to the center of momentum frame. This will be where the collision is least energetic, and the total energy in this frame is the minimum necessary. \\

In the lab frame we have

\begin{align*}
\left(\begin{matrix}
E_\gamma/c \\
E_\gamma/c \\
0 \\
0
\end{matrix}\right) +
\left(\begin{matrix}
Mc \\
0 \\
0 \\
0
\end{matrix}\right) =
\left(\begin{matrix}
E_\gamma'/c \\
p_{\gamma x}' \\
p_{\gamma y}' \\
p_{\gamma z}'
\end{matrix}\right) +
\left(\begin{matrix}
E_M'/c\\
p_{Mx} \\
p_{My} \\
p_{Mz}
\end{matrix}\right) +
\left(\begin{matrix}
E_+/c \\
p_{+x} \\
p_{+y} \\
p_{+z}
\end{matrix}\right) +
\left(\begin{matrix}
E_-/c \\
p_{-x} \\
p_{-y} \\
p_{-z}
\end{matrix}\right)
\end{align*}

where the \(\pm\) refer to the particle and antiparticle respectively. Symmetry requires \(p_+ = -p_-\) and \(E_+ = E_-\) so that the above becomes
\begin{align*}
\left(\begin{matrix}
(E_\gamma+Mc^2)/c \\
E_\gamma/c \\
0 \\
0
\end{matrix}\right)&=
\left(\begin{matrix}
(E_\gamma'+E_M' +2E_+)/c \\
p_{\gamma x}'+p_{Mx}' \\
0 \\
0
\end{matrix}\right)
\equiv \left(\begin{matrix}
E'/c \\
p' \\
0 \\
0
\end{matrix}\right)
\end{align*}

The invariant interval of the total reactant 4-momentum in the lab frame is \(2E_\gamma M + M^2 c^2\). 4-momentum is conserved, so that the total product 4-momentum in the CM frame must also have this same invariant. The 3-momentum in the CM frame is zero, so if \(E^*\) is the CM frame energy we must have

\[E_\gamma M + M^2 c^2 = {E^*}^2/c^2
\] 

If not for the product photon we could find \(E^*\) right away by noting that in the CM frame, the only contribution to the total energy would be from mass energy of the product particles. Instead, the above becomes the underconstrained

\[E_\gamma M + M^2 c^2 = {(E_\gamma^{'*} + \sqrt{3M}c)}^2/c^2
\] 

So this is as far as we can take the problem unless we specify something else about it, such as the angle between the reactant and product photons.




%We transform from the lab frame to the CM frame by operating with the \(+x\) lorentz boost:
%
%\begin{align*}
%\left(\begin{matrix}
%\gamma & -\beta\gamma &0&0\\
%-\beta\gamma &\gamma &0&0 \\
%0&0&1&0 \\
%0&0&0&1
%\end{matrix}\right)
%\end{align*}
%
%This yields \\
%\begin{align*}
%\left(
%\begin{array}{c}
% \gamma  \left(c^2 M-(\beta -1) E_{\gamma }\right) \\
% \gamma  \left(\beta  c^2 (-M)-(\beta -1) E_{\gamma }\right)
%   \\
% 0 \\
% 0 \\
%\end{array}
%\right)=
%\left(
%\begin{array}{c}
% \gamma  \left(E_{\gamma }'-\beta  c
%   \left(p_{\text{Mx}}'+p_{\text{$\gamma $x}}'\right)+2
%   E_++E_M'\right) \\
% c \gamma  \left(-\frac{\beta  \left(E_{\gamma }'+2
%   E_++E_M'\right)}{c}+p_{\text{Mx}}'+p_{\text{$\gamma
%   $x}}'\right) \\
% 0 \\
% 0 \\
%\end{array}
%\right)
%\end{align*}
%
%\[\left(
%\begin{array}{c}
% \gamma  \left(c^2 M-(\beta -1)
%   E_{\gamma }\right) \\
% \gamma  \left(\beta  c^2
%   (-M)-(\beta -1) E_{\gamma
%   }\right) \\
% 0 \\
% 0 \\
%\end{array}
%\right)
%\]

\subsubsection*{4}

This problem has been solved in lecture twice now by the method of eigenfunction expansion, so instead I'll attempt this using symmetry and the divergence theorem in n dimensions.

Applying the divergence theorem to \(\nabla\phi\),

\begin{align*}
\oint\limits_{\partial V} (\nabla\phi \cdot \hat{n}) \mathrm{d (\partial V)} &= \int\limits_{V} (\nabla \cdot \nabla\phi)\mathrm{dV}\\
&= \int\limits_{V} \nabla^2\phi\mathrm{dV}\\
&= \int\limits_{V} K_4 Q \delta(0)\mathrm{dV}\\
&= K_4Q
\end{align*}

From here, the only assumption we need is that \(\nabla\phi\) is spherically symmetric. Then let V be the 4-sphere of radius r; the integral over \(\partial V\) becomes

\[S(r)|\nabla\phi(r)| = K_4 Q
\]

where \(S(r)\) is the surface volume of the 4-sphere of radius \(r\). We can find \(S(r)\) by first considering the interior hyper volume of the 4-ball of radius \(r\) and then differentiating with respect to \(r\).

The basic approach is to generalize spherical coordinates to four dimensions. Proceeding by analogy with spherical  coordinates, let \(r^2 = w^2 + z^2 + y^2 + x^2\), and let \( \theta_w \in [0,\pi]\) be the angle to the \(w\) axis. Then \(w = r \cos \theta_w \), and the definition of \(r\) becomes

\begin{align*}
 r^2\cos^2\theta_w + z^2 + y^2 + x^2 &= r^2 \\
 \implies z^2 + y^2 + x^2 &= r^2(1-\cos^2\theta_w) \\
 &= r^2\sin^2\theta_w
\end{align*} 

If we take \(z = r\sin\theta_w\cos\theta_z\), \(\theta_z \in [0,\pi]\), notice that by substituting into the above we have

\begin{align*}
y^2 + x^2  &= r^2\sin^2\theta_w(1-\cos^2\theta_z) \\
&= r^2\sin^2\theta_w\sin^2\theta_z
\end{align*}

Repeating for \(y = r\sin\theta_w\sin\theta_z\cos\theta_y \) then yields \(x^2 = r^2\sin^2\theta_w\sin^2\theta_z\sin^2\theta_y\). If \(\theta_y \in [0,\pi]\) then \(x = \pm r\sin\theta_w\sin\theta_z\sin\theta_y \). We can get rid of the \(\pm\) though by just choosing instead that \(\theta_y \in [0,2\pi]\).

In order to compute the volume integral, we require the Jacobian determinant J:

\begin{align*}
	\left|\begin{matrix}
	\frac{\partial x}{\partial r} & \frac{\partial x}{\partial \theta_w} & \frac{\partial x}{\partial \theta_z} &\frac{\partial x}{\partial \theta_y} \\[6pt]
	\frac{\partial y}{\partial r} & \frac{\partial y}{\partial \theta_w} & \frac{\partial y}{\partial \theta_z} &\frac{\partial y}{\partial \theta_y} \\[6pt]
	\frac{\partial z}{\partial r} & \frac{\partial z}{\partial \theta_w} & \frac{\partial z}{\partial \theta_z} &\frac{\partial z}{\partial \theta_y} \\[6pt]
	\frac{\partial w}{\partial r} & \frac{\partial w}{\partial \theta_w} & \frac{\partial w}{\partial \theta_z} &\frac{\partial w}{\partial \theta_y} \\
	\end{matrix}\right|
\end{align*}

which is straightforward to compute (especially with software assistance...)

\[J=r^3\sin^2\theta_w\sin\theta_z
\]

The volume of the 4-ball is then
\begin{align*}
V(r) &= \int\limits_{0}^{r}\int\limits_{0}^{2\pi}\int\limits_{0}^{\pi}\int\limits_{0}^{\pi}J\ d\theta_w d\theta_z d\theta_y dr = \frac{1}{2}\pi^2r^4
\end{align*}

So \(S(r) = 2\pi^2r^3 \). Since \(K_4\) is a space independent constant we require \(|\nabla \phi(r)| \propto r^{-3}\). Using again that \(\phi\) is spherically symmetric, we can write

\begin{align*} 
\nabla\phi &= \left(\begin{matrix}
1 \\ 1\\ 1 \\ 1\\
\end{matrix}\right)\phi ' (r) \\[6pt]
|\nabla\phi| &= \phi ' (r)
\end{align*} 

leaving us with \(\phi \propto r^{-2}\). If we want \(\phi = Q/r^2 \) then \(|\nabla \phi| = 2Q/r^3\) and \(K_4 = 4\pi^2\). 

\end{document}