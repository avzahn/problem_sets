\documentclass[12pt]{article}
\usepackage{amsmath}
\usepackage{braket}
\usepackage{bbold}
\usepackage{mathrsfs}

\author{Alex Zahn}
\title{Phys 203a Problem Set II}
\date{}

\begin{document}
\maketitle
\subsection*{1}

The \(i^{th}\) Euler-Lagrange equation for this \(L\) (correcting for the missing factor of \(Q\) in the \(A\) term) is

\begin{align*}
0 &= -Q \frac{\partial \phi}{\partial x_i} + \frac{Q}{c}\sum_j\frac{\partial A_j}{\partial x_i}\dot{x}_j - \frac{d}{dt}\left( M\dot{x}_i  + \frac{Q}{c} A_i\right)\\
&= -Q \frac{\partial \phi}{\partial x_i} + \frac{Q}{c}\sum_j\frac{\partial A_j}{\partial x_i}\dot{x}_j - M\ddot{x}_i  - \frac{Q}{c} \sum_j \frac{\partial A_i}{\partial x_j}\dot{x_j} - \frac{Q}{c}\dot{A}_i\\
&= -M\ddot{x}_i + QE_i + \frac{Q}{c}\sum_j \left( \frac{\partial A_j}{\partial x_i}\dot{x}_j-\frac{\partial A_i}{\partial x_j}\dot{x}_i\right) \\
&= -M\ddot{x}_i + QE_i + \frac{Q}{c} (\dot{x} \times (\nabla \times A))_i \\
\end{align*}

which yields

\[ M\ddot{x} = QE + \frac{Q}{c}\dot{x} \times B
\]

\subsection*{2}

If the charge density \(\rho_q\) is given, we can find \(J\) by imposing local conservation of charge:

\[\frac{\partial \rho_q}{\partial t} + \nabla \cdot J = 0
\]

It's easy to see by direct substitution into the above that if 

\[\rho_q = Q\delta^3(x_0 - x)\]

we can satisfy this condition with

\[J = Q\dot{x}\delta^3(x_0-x)
\]

Now consider the nonrelativistic lagrangian for a point charge,

\[ L = \frac{1}{2}m\dot{x}^2 + Q(\frac{1}{c}\dot{x}\cdot A -\phi)
\]

Intuition strongly suggests we can generalize to an arbitrary charge distribution by associating \(\dot{x}\) with the current density \(J\), \(Q\) with the charge density \(\rho_q\), and \(m\) with the mass density \(\rho_m\). Invoking the ancient rite of dimensional analysis, we propose a lagrangian density

\begin{align*}
\mathscr{L} &= \frac{1}{2}\frac{\rho_m}{\rho_q^2}J^2  +\frac{1}{c}J\cdot A - \rho_q \phi\\
L &= \int \mathrm{d}^3x \mathscr{L} \\
\end{align*}

where \(\rho_m\) is the mass density, which for a point particle would be
\[ \rho_m = m\delta^3(x_0 - x)
\]

Inserting the expressions for \(\rho_q\), \(\rho_m\), and \(J\) belonging to a point charge, we recover the original point charge lagrangian. The resulting lorentz force law is
\[ \frac{\rho_m}{\rho_q}\dot{J} = \rho_qE + J \times B/c
\]

\subsection*{3}

It's easy to see that if we take \(E=0\), the only acceleration on the mass is orthogonal to its velocity due to the \(\dot{x}\times B\) term, so that energy is conserved if electric field vanishes.

\subsection*{4}

Without loss of generality, we can take the constant \(B\) field to be along the \(\hat{z}\) axis. The equation of motion becomes

\begin{align*}
\begin{cases}
m\ddot{x}_1 = QE_1 + QB_0\dot{x}_2/c \\
m\ddot{x}_2 = QE_2 - QB_0\dot{x}_1/c \\
m\ddot{x}_3 = QE_3
\end{cases}
\end{align*}

The solution for \(x_3\) given \(x(0)=\dot{x}(0) =0\) is seen by inspection:

\[x_3 = \frac{1}{2}QE_3t^2
\]

The rest of the solution is less trivial:

\begin{align*}
x_1 &= E_2t/B_0 + \frac{mc^2}{B_0^2Q} \left( E_1  - E_2\sin\frac{B_0Qt}{mc} - E_1\cos\frac{B_0Qt}{mc}\right)\\
x_2 &= E_1t/B_0 + \frac{mc^2}{B_0^2Q} \left( E_2  - E_2\cos\frac{B_0Qt}{mc} + E_1\sin\frac{B_0Qt}{mc}\right)\\
\end{align*}

\subsection*{5}

The equation of motion is now

\begin{align*}
\begin{cases}
m\ddot{x}_1 = 0 \\
m\ddot{x}_2 =  QB_0\cos(\frac{\omega t}{c}-kx_3)\dot{x}_3/c \\
m\ddot{x}_3 =  -QB_0\cos(\frac{\omega t}{c}-kx_3)\dot{x}_2/c
\end{cases}
\end{align*}

If the particle starts from rest, we can see that there won't be any acceleration to move it from its initial position. This is lucky, because this system looks more than a little difficult to solve in general.


\subsection*{6}

See Problem Set 3 for a complete derivation of the generators of rotation for \(\mathbb{R}^3\). In general though, for any given space \(S\), the generator of rotation about an arbitrary \(\hat{n}\) is \(J \cdot \hat{n}\), where \(J = (J_1,J_2,J_3)\)  with \(J_i:S \rightarrow S\) and \([J_i,J_j]=i\epsilon_{ijk}J_k\) 

\end{document}